\chapter{Time and Global States}
\begin{multicols*}{2}

\noindent Although one of the property of distributed system is no globally shared clock, many distributed applications still depend on timing.

\section{Synchronizing Physical Clock}

\noindent Computers have their own physical clocks that deviate from one another. Universal Coordinated Time (UTC) is commonly used reference clock.\\

\noindent \emph{Clock drift}: clock count time at different rates.\\

\noindent \emph{Drift rate}: the change in offset between a clock and a perfect reference clock per unit of time measured by the reference clock.

\subsection{Cristian's Method}

\noindent The algorithm:

\begin{enumerate}
  \item $A$ requests the time of $B$
  \item $B$ replies with the time value $t$
  \item $A$ records the total RTT, $T_{\text{round}}=t_2 - t_1$
  \item $A$ sets its clock to $t + T_{\text{round}} / 2$
\end{enumerate}

\noindent To calculate the accuracy of Cristian's Method, we need to consider 2 cases:
\begin{enumerate}
  \item When the transmission time of reply message approaches $0$, $B$'s clock reading is $t$.
  \item When the transmission time of reply message approaches $T_{\text{round}}$, $B$'s clock reading is $t+ T_{\text{round}}$.
\end{enumerate}

\noindent When $A$ receives the reply, the time reading on $B$ should be in $[t,t+ T_{\text{round}}]$. $A$ is accurate within bound $T_{\text{round}} / 2$

\subsection{Berkeley Algorithm}

\noindent The algorithm:
\begin{enumerate}
  \item Master periodically polls slaves
  \item Slaves send their clock readings back
  \item Master evaluates the local times of slaves by observing the RTT
  \item Master averages the values obtained and its own clock reading
  \item Master sends the amount of adjustment to each slave
\end{enumerate}

\noindent We use average clock reading because it cancels out individual clocks' tendencies to run fast or slow. 

\subsection{Network Time Protocol}

\noindent Cristian's method and Berkeley algorithm use centralized design. Network Time Protocol (NTP) is used to distribute time information over the Internet. \\

\noindent NTP has a network of servers structured hierarchically into a synchronization subnet. This setup is fault-tolerance because servers can reconfigure themselves if someone becomes unreachable. 

\subsubsection{Analysis of Accuracy}
\begin{center}
\begin{tikzpicture}[>=stealth',thick,
commentl/.style={text width=0.7cm, align=right},
commentr/.style={commentl, align=left},]
\node[] (init) {Server B};
\node[right=1cm of init] (recv) {Server A};

\draw[->] ([yshift=-.5cm]init.south) coordinate (send1) -- ([yshift=-.7cm]send1-|recv) coordinate (send1done) node[pos=.3, above, sloped] {};

\draw[->] ([yshift=-.3cm]send1done) coordinate (reply1) -- ([yshift=-.7cm]reply1-|init) coordinate (reply1done) node[pos=.3, above, sloped] {m};

\draw[->] ([yshift=-1cm]reply1done) coordinate (send2) -- ([yshift=-.7cm]send2-|recv) coordinate (send2done) node[pos=.3, above, sloped] {m'};

\draw[->] ([yshift=-.3cm]send2done) coordinate (reply2) -- ([yshift=-.7cm]reply2-|init) coordinate (reply2done) node[pos=.3, above, sloped] {};

\draw[thick, shorten >=-1cm] (init) -- (init|-reply2);
\draw[thick, shorten >=-1cm] (recv) -- (recv|-reply2);

\node[commentr, right =1mm of reply1] {$T_{i-3}$};
\node[left = 1mm of reply1done.west, commentl]{$T_{i-2}$};
\node[left = 1mm of send2.west, commentl]{$T_{i-1}$};
\node[commentr, right =1mm of send2done] {$T_{i}$};
\end{tikzpicture}
\end{center}

\noindent Time between sending of $m$ and receipt of $m'$ is $$T_i - T_{i-3}$$
\noindent Time between receipt of $m$ and sending of $m'$ is $$T_{i-1} - T_{i-2}$$
\noindent The total transmission time of $m$ and $m'$ is $$(T_i - T_{i-3}) - (T_{i-1} - T_{i-2})$$
\noindent The transmission time of $m'$ is $$[0,(T_i - T_{i-3}) - (T_{i-1} - T_{i-2})]$$
\noindent If transmission time of $m'$ approaches $0$, when A receives $m'$ B clock reading is $$T_{i-1}$$
\noindent If transmission time of $m'$ approaches $(T_i - T_{i-3}) - (T_{i-1} - T_{i-2})$, when A receives $m'$ B clock reading is $$T_{i-1} + (T_i - T_{i-3}) - (T_{i-1} - T_{i-2}) = T_i - T_{i-3} + T_{i-2}$$
\noindent If A want to synchronize with B as accurately as possible, A should set its clock to $$\frac{1}{2}(T_{i-1} + (T_i - T_{i-3} + T_{i-2})$$

\noindent The accuracy is $$\pm \frac{1}{2}(T_{i-1} - (T_i - T_{i-3} + T_{i-2})$$

\section{Causal Ordering and Logical Clocks}

\noindent We refer the state of process $p_i$ and as $s_i$, and there are $i=1\ldots N$ number of processes.\\

\noindent Process takes a series of actions, i.e. sending a message, receiving a message, or transforms the state of the process.\\

\noindent An event is the occurrence of a single action.

\subsection{Causal Ordering}

\noindent Events can be ordered based on cause-and-effect. Event within a single process $p_i$ can be placed in a unique total ordering. The event of sending the message occurs before the event of receiving the message. 

\begin{center}
\begin{tikzpicture}[>=stealth',thick,
roundnode/.style={circle, draw=black, thick, minimum size=0.7cm},
]
\node[] (p1) {$p_1$};
\node[below=1cm of p1] (p2) {$p_2$};
\node[below=1cm of p2] (p3) {$p_3$};

\node[roundnode](a)[right = 1mm of p1.east]{a};
\node[roundnode](b)[right = 12mm of a.east]{b};

\node[roundnode](c)[right = 25mm of p2.east]{c};
\node[roundnode](d)[right = 12mm of c.east]{d};

\node[roundnode](e)[right = 12mm of p3.east]{e};
\node[roundnode](f)[right = 50mm of p3.east]{f};

\node[right = 60mm of p1](d1){};
\node[right = 60mm of p2](d2){};
\node[right = 60mm of p3](d3){};

\draw[-] (a) -- (b) node [pos=0.5, above, align=center] {$a\rightarrow b$};
\draw[->] (b) -- (c) node [pos=0.5, right, align=center] {$b\rightarrow c$};
\draw[-] (c) -- (d) node [pos=0.5, above, align=center] {$c\rightarrow d$};
\draw[->] (d) -- (f) node [pos=0.5, right, align=center] {$d\rightarrow f$};
\draw[-] (e) -- (f) node [pos=0.5, above, align=center] {$e\rightarrow f$};
% Dummy 
\draw[-] (b) -- (d1);
\draw[-] (p2) -- (c);
\draw[-] (d) -- (d2);
\draw[-] (p3) -- (e);
\draw[-] (f) -- (d3);

\end{tikzpicture}
\end{center}

\noindent From the figure above, we can deduce that $a$ happens before $f$, $a\rightarrow f$, and $a$ is parallel to $e$, $e\|e$ 

\subsection{Lamport's Logical Clocks}

\subsection{Vector Clocks}

\section{Global States}

\subsection{Consistent Cut}

\subsection{Recording a Snapshot}

\subsection{Chandy and Lamport Algorithm}

\section{Distributed Debugging}

\subsection{Lattice of Consistent Global States}

\subsection{Observing Consistent Global States}

\end{multicols*}
